\documentclass[10.8pt,letterpaper]{article}

% Encoding & language
\usepackage[utf8]{inputenc}
\usepackage[T1]{fontenc}
\usepackage[french]{babel}

% Layout
\usepackage[margin=1.8cm]{geometry}
\usepackage{multicol}
\setlength{\columnsep}{0.9cm}
\usepackage{enumitem}
\setlist[itemize]{noitemsep, topsep=2pt, leftmargin=1.1em}
\setlist[description]{font=\normalfont\bfseries,leftmargin=1.1em, labelsep=0.4em}
\usepackage{parskip}

% Visuals
\usepackage{titlesec}
\titleformat{\section}{\large\bfseries}{}{0pt}{}
\usepackage{tcolorbox}
\tcbset{colback=gray!10!white, colframe=black!40, boxrule=0.3pt, arc=2mm, left=2mm, right=2mm, top=1mm, bottom=1mm}

% Header
\newcommand{\GameTitle}{Stickman Brawl}
\newcommand{\Subtitle}{One-Pager}
\newcommand{\Version}{v2.0 — \today}

\begin{document}
\begin{tcolorbox}
  {\Large \textbf{\GameTitle}}\\[-2pt]
  {\normalsize \Subtitle} \hfill {\footnotesize \Version}
\end{tcolorbox}

\begin{multicols}{2}

\section*{Pitch}
\textbf{Stickman Brawl} est un jeu de combat d'arène \textbf{multijoueur local} en \textbf{vue de côté}, inspiré de la série Super Smash Bros. Les stickmen se battent avec attaques et combos directionnels pour \textbf{éjecter leurs adversaires hors de l’arène}. Prototype jouable \textbf{2 joueurs locaux} d'ici \textbf{19 oct. 2025}.

\section*{Piliers}
Lisibilité / Intensité / Maîtrise des contrôles / Rounds rapides.

\section*{Boucle de jeu}
\begin{itemize}
  \item \textbf{Préparation} : lancement rapide, choix couleur/personnage.
  \item \textbf{Combat} : \item \textbf{Combat} : affrontement dynamique 1v1 dans une arène ouverte, où l’objectif est d’éjecter l’adversaire.
  \item \textbf{Fin de match} : l’adversaire est éjecté hors de l’écran.
  \item \textbf{Relance} : possibilité de relancer immédiatement ou revenir au menu principal.
\end{itemize}

\section*{Mécaniques clés}
\begin{itemize}
  \item \textbf{Déplacements} : marche, course, saut et double saut.
  \item \textbf{Dashes} : au sol et en l’air pour mobilité et esquive.
  \item \textbf{Attaques directionnelles} : légères (rapides) et lourdes (chargeables).
  \item \textbf{Lancer} : saisir et projeter un adversaire.
  \item \textbf{Victoire} : éjecter l’adversaire hors de l’arène (KO).
\end{itemize}

\section*{Contrôles (Xbox)}
\begin{description}
  \item[Stick gauche] Déplacements / Orientation
  \item[A] Attaque lourde (maintenir pour charger)
  \item[B] Attaque légère (directionnelle selon le stick)
  \item[RB/LB] Saisie
  \item[RB/LB] Bouclier / Esquive
  \item[X/Y] Saut
  \item[Start] Menu pause (Reprendre / Relancer / Quitter)
\end{description}

\section*{Contenu visé (réaliste)}
Prototype minimal mais complet :  
\begin{itemize}
  \item \textbf{Joueurs} : 2 stickmen jouables.
  \item \textbf{Arène} : plateforme centrale et bords ouverts.
  \item \textbf{Mécaniques} : déplacements, esquive, bouclier, double saut, attaques légères/lourdes, saisie.
  \item \textbf{Feedback} : hitstop, knockback progressif, sons percussifs simples.
  \item \textbf{HUD} : compteur de dégâts (pourcentage), chrono, menu pause.
  \item \textbf{Défaite} : éjection hors de l’écran = KO.
\end{itemize}

\section*{Public cible \& plateformes}
\begin{itemize}
  \item \textbf{Public} : jeunes adultes et amateurs de jeux de combat locaux.
  \item \textbf{Âge recommandé} : 10+ (violence cartoonesque, sans sang).
  \item \textbf{Plateforme} : PC Windows, support natif manette Xbox.
  \item \textbf{Perf cible} : 60 FPS stables sur machine de milieu de gamme.
\end{itemize}

\section*{Style visuel \& audio}
\begin{itemize}
  \item \textbf{Visuel} : stickmen, couleurs au choix.
  \item \textbf{Indicateurs} : pourcentage de dégâts visible sous forme stylisée.
  \item \textbf{Arène} : plateformes simples, arrière-plan neutre, lisibilité prioritaire.
  \item \textbf{Audio} : sons d’impact clairs et secs; musique rythmée et courte en boucle.
\end{itemize}

\section*{Tech \& prod}
Unreal Engine 5.6 (Blueprint) • Dépôt Git • Documentation \LaTeX{}.

\section*{Risques \& parades}
\begin{itemize}
  \item \textbf{Précision du knockback} : itérations régulières pour équilibrer vitesse et force.
  \item \textbf{Lisibilité des collisions} : silhouettes contrastées, feedback visuel clair.
  \item \textbf{Scope creep} : se limiter à 2 joueurs, 1 arène et un set réduit de coups.
\end{itemize}

\section*{USP}
\begin{itemize}
  \item \textbf{Simplicité stickman} : lisible et stylisé, facile à suivre.
  \item \textbf{Knockback progressif} : tension croissante à chaque coup.
  \item \textbf{Combat pick-up-and-play} : prise en main rapide, relance instantanée.
  \item \textbf{Multijoueur local immédiat} : fun accessible à 2 joueurs dès le prototype.
\end{itemize}

\end{multicols}
\end{document}
